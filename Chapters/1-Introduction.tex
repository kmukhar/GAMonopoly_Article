\clearpage
\chapter{Introduction}\label{chap:intro}
Games can be used to study many different real world problems. In the
classic monograph by John Von Neumann and Oskar Morgenstern, \emph{Theory of
Games and Economic Behavior}, it was applied to economics and economic behavior
\cite{neumann1944theory}.

Early work by mathematicians studying card and dice games led to the development
of statistics and probability \cite{hald1990history,rudas2008handbook}. Games
can be studied from a sociological point of view, looking at how social behavior
and human interactions affect and are affected by game play
\cite{fararo1992meaning}. As mentioned previously, games can be studied as a
model of economics and decision making \cite{neumann1944theory}. A great deal of
artificial intelligence research deals with developing AI game players and
finding optimum game strategies \cite{russell2010artificial}.

\section{Objective}

Early research into AI and games tended to focus on developing algorithms for
searching the space of legal moves. This research led to such techniques as the
Min-Max algorithm~\cite{neumann1944theory} and the A* search
algorithm~\cite{Hart_Nilsson_Raphael_1968,Hart:1972:LFB:1056777.1056779}. The
Min-Max algorithm was originally developed by Von Neumann as a way to develop
the best move in a 2-person zero-sum game with complete information. Although
originally developed for a 2-person game, the min-max theorem has been extended
to cover any number of players. The A* algorithm is a method to find the least
cost path in traversing a search space or graph.

With the development of genetic algorithms and evolutionary computation,
re\-search\-ers in this area have investigated many games in attempts to show
that optimal strategies for games can be evolved. Much of that research has
focused on two player deterministic games like The Prisoner's
Dilemma~\cite{Flood1958} and Tic-Tac-Toe. Evolutionary computation techniques
have been used with good success for these types of problems.

There are other classes of games, however, such as multi-player
non-deterministic games where the search space is potentially quite large and
the cost function could have multiple local and global optima. Any game that
relies on some form of non-determinism (i.e., rolling dice, shuffling cards,
etc.) falls into this category. Evolutionary computation could be an effective
approach to these types of problems because it provides the capability to
traverse a large search space and find optimal strategies.

This paper proposes a thesis project to investigate using evolutionary
computation to evolve game strategies for these types of games. In general, the
proposed project will study whether genetic algorithms can be used to develop
strategies for non-deterministic games or games with incomplete knowledge.

\section{Organization of Thesis}

The thesis project proposed here will investigate evolutionary computing as a
means to develop game playing strategies. Since the primary subject
matter of this thesis is genetic algorithms, we start in Chapter
\ref{chap:geneticAlgorithms} with a brief overview of genetic algorithms. 

Chapter \ref{chap:background} provides a survey of some previous research into
evolutionary computation and games that will be applied in this thesis. We will
look at both practical and theoretical research that applied evolutionary
computing to various games.

The game known as Monopoly was chosen as the subject of this research. Chapter
\ref{chap:monopoly} provides details about Monopoly, focusing on the rules of
the game.

The details of how a genetic algorithm will be used in this research is provided
in Chapter \ref{chap:algorithm}. We look at various formats for the genome,
along with different options for measuring genome fitness, and the process for
propagating a population.

Finally, The results of this research are detailed in Chapter
\ref{chap:results}.
