\clearpage
\chapter{Background Research}\label{chap:background}

\section{Overview}

Research in applying evolutionary computation to games\footnote{The term
``game'' is highly overridden. Throughout this chapter (and the entire thesis)
the term generally refers to board games or other similar types of competitions
that are played between two or more persons} has been performed almost since the
study of evolutionary computing began. Holland, in his seminal work
\emph{Adaptation in Natural and Artificial Systems}, discussed applications of
genetic algorithms to games~\cite{holland1975adaptation,Holland1992}. Holland
showed that the search space of many games quickly grew to a size that is not
computationally tractable. He then discussed how an artificial adaptive system
(i.e., evolutionary computation) could be used to efficiently search the game
space and realize an optimal strategy.

In Chapter \ref{chap:algorithm} the design of the genetic algorithms used for
this research is presented. The decisions made for those algorithms have a basis
in research that applied genetic algorithms to the study of various games. In
the remainder of this chapter we review some of that research.

\section{Deterministic zero sum games}

One fundamental type of game is the deterministic two player game with perfect
information. To be deterministic means that the game has no element of
randomness or chance in it. Two player obviously means that only two players are
involved in the game. (Using only two players is done to make computations easy,
but the results are easily extended to games with more than two players.) A two
player game involves a gain or payoff to one player and a loss to the other
player. The gain and loss are not necessarily equal; when the gain to one player
is equal to the loss experienced by the other player in the game, the game is
known as a zero sum game. In its simplest form the payoff is a binary win/loss
result: one player wins and the other player loses. Finally, for a game to
involve perfect information means that every player in the game knows everything
about every the other player's state and future possible moves.

\subsection{The Prisoner's Dilemma}

One of the most studied deterministic two player games is the two person game
known as the Prisoner's Dilemma, a game in which two players can either
cooperate or defect, without the ability to communicate with each other. The
payoff matrix for this game is shown in Table \ref{table-pdpayoff}. This game is
not a zero sum game. 

\begin{table}[h]
\caption{Payoff matrix for Prisoner's Dilemma}
\begin{center}
\begin{tabular}{|l|c|c|}
\hline
\multicolumn{1}{|l|}{\raisebox{-1.50ex}[0cm][0cm]{Decision\!}}
& \multicolumn{1}{|c|}{Cooperate}
& \multicolumn{1}{|c|}{Defect} \\ \hline
Cooperate &  R,R & S,T \\ \hline 
Defect    &  T,S & P,P  \\ \hline \hline
Notes & \multicolumn{2}{|c|}{\(T > R > P > S\)} \\ 
      & \multicolumn{2}{|c|}{\(2R > T + S\)} \\ \hline
\end{tabular}
\label{table-pdpayoff}
\end{center}
\end{table}

In a single iteration of this game, the best strategy for both players is to
defect~\cite{nash50,Nash1951}. Suppose player 1 chooses to cooperate;
looking at the payoffs, this player sees that regardless of which strategy the
opponent chooses, player 1 gets a better score by switching to defect. The same
reasoning applies to player 2.

However, studies have shown that when the game is played repeatedly (known as
the Iterated Prisoner's Dilemma game), a strategy of cooperation will
spontaneously evolve within a
population~\cite{Axelrod1984,Nowak1993,DBLP:conf/cig/QuekG07,Mittal2009,WangTao2010}.
This game is different from the others presented below in that there is an
objective measure of fitness for this game. For Nowak and Sigmund, ``Strategies
with higher payoffs produce more offspring.''~\cite{Nowak1993} Similarly for
Quek and Goh, ``\ldots fitness assignment is performed after each tournament and
is given by the payoffs accumulated throughout the game
play.''~\cite{DBLP:conf/cig/QuekG07} Mittal and Deb studied the Prisoner's
Dilemma as ''\ldots a bi-objective optimization problem of maximizing the
player's own score and simultaneously minimizing the opponent's score'' where
the score was the sum of the player's payoff over the set of games. Wang et al.
also used the sum of payoffs as a fitness measure~\cite{WangTao2010}.

In other games, having an objective fitness measure is not always possible. In
some games the only thing that can be measured is which player won the game and
which lost; but the winner does not necessarily have high fitness and the loser
does not necessarily have low fitness. A mediocre player could get a large
number of wins if many of the opponents are worse players. In these cases, a
competitive fitness function is used. In the remainder of this section we look
at several studies that used competitive fitness functions.

\subsection{Tic-Tac-Toe}

The game of Tic-Tac-Toe is a relatively simple two-player competitive game
played on a 3x3 grid. Players alternately place a symbol, (usually one player
plays X and the other plays O) on an open grid space; the winner is the player
that manages to create a straight line (vertically, horizontally, or diagonally)
of three symbols. Fogel showed that genetic algorithms could be used to evolve
neural networks to play the game~\cite{Fogel1993}.

Fogel played a population of 50 randomly generated neural networks against a
rule-based opponent which also had a small probability of making a random move.
Each neural network played 32 games against the rule-based player. A competitive
fitness function was used in which the neural network was awarded +1 for
winning, -10 for losing, and 0 for a draw. Because of the element of randomness,
a neural network with a high score did not necessarily have a perfect strategy.
That is, a neural network could have a high score because the opponent made
several bad moves over the 32 games. So, before propagating the population,
Fogel ran a further competition where a neural network played 10 other randomly
chosen networks. Neural networks that scored high in the second competition were
allowed to reproduce to the next generation.

In this example, the basic elements of a competitive fitness functions are
present: 1) individuals in the population 2) compete against some other player
and 3) are awarded points for winnning, losing, or drawing. In this study, the
individuals in the population competed against static players from outside the
population. However, individuals in a population can also compete against each
other in a process known as coevolution.

\subsection{Checkers}

In Checkers each player has 12 pieces that can move diagonally on a board
divided into an 8x8 grid of squares. If one player's piece is adjacent to the
opponent's piece, and there is a blank square on the opposite side of the
opponent's piece, the player can jump that piece and remove the opponent's piece
from the board. The winner is the player who removes all of the opponent's
pieces from the board. Fogel and Chellapilla used evolutionary computing  to
evolve neural networks that could generate moves for the
game~\cite{Fogel2000Anaconda,journals/tec/ChellapillaF01}. 

Interestingly, Fogel's genetic algorithm did not use any reproduction operators
on the genome. The genome was evolved by picking the fittest genomes to carry
over to the new generation, with new genomes created by mutation only.

Fogel and Chellapilla used a competitive fitness function for their research.
A population consisted of 15 parents from which 15 children were created by
mutation. Each neural network in the population then played five games against
five other randomly chosen neural networks from the population. Wins, losses,
and draws were awarded +1, -2, or 0 points respectively. After all games had
been played, the top 15 neural networks by total score were copied into the next
generation and the process was repeated. Again, neural networks with high scores
were not necessarily the best players, since a high score could be obtained by a
player with poor strategy if the opponents of the player were all worse players.
To validate their results, Fogel and Chellapilla took the best evolved neural
network and played it against human players over the Internet. After 250
generations, it played just below expert level, and after 840 generations their
evolved ANN player was able to play at Expert level, well above the average
human player.

This is an example of coevolution. In this work the competitive fitness function
consists of individuals competing against other individuals in the same
population. However, each individual only competes consists of a subset of the
population.

\subsection{Othello}

Othello (also known as Reversi) is a two-player game where players take turns
placing colored tokens on the board. The winner is the player who can get the
most tokens of his color on the board. Play is complicated by the fact that a
player can only place a token in a way that ``captures'' the opponent's tokens.
This is done by placing a token adjacent to an opponent's token such that at
least one of the opponent's tokens lies between the new token and one of the
already placed tokens. When a player captures a line of the opponent's tokens,
all of the opponent's tokens in that line change to the capturing player's
color. Chong et al., investigated evolving the evaluation function of a neural
network to create an ANN that could play Othello~\cite{ChongTW05}.

The population consisted of 10 parent neural networks, with 10 children created
through mutation. Each of the 20 individuals in the population then played 5
games against other neural networks chosen randomly from the population. Wins
were awarded five points, losses received zero points, and draws resulted in
each network getting one point. The top ten scoring individuals were propagated
to the next generation as parents. At each generation, a second evaluation was
performed by having the best neural networks play against a computer player that
used a deterministic evaluation function to generate moves. These secondary
competitions were only used to evaluate the ability of the neural networks, and
were not used to guide the evolution. As the population evolved, competition
against the computer player showed that the neural networks were indeed playing
at higher levels, achieving master level at about the 800th generation.

This example of coevolution is very similar to the previous example.

\subsection{State Space Complexity}
We could continue to cite other research into genetic algorithms and games, and
will make a brief mention below. But at this point it will be instructive to
look at the state space complexity of games. The first four games cited above
were introduced from the least complex to the more complex.

The state space complexity of these and other games is well known and can be
easily found (or calculated). Allis, for example, provides the following
estimates for several games including some of the ones~\footnote{Allis looked at
games from the Computer Olympiad (see
\url{http://www.grappa.univ-lille3.fr/icga/competition.php?id=3} which does not
include The Prisoner's Dilemma. However, the state space for The Prisoner's
Dilemma is easy to calculate. With a single iteration, each player has two
possible moves, so the state space complexity is \(2\cdot2=4\). If iteration is
allowed, then with two iterations each player has four moves and the state space
complexity is 16. Three iterations gives a state space complexity of 64. In
general, the state space complexity is \(2^{2n}\) where \(n\) is the number of
iterations.} discussed above~\cite{Allis1994}. Tic-Tac-Toe has an estimated
state space complexity of \(\approx10^{3.7}\). The state space complexity for
Checkers is \(\approx10^{18}\). As might be expected, Othello is more
complicated with a state space complexity of \(\approx10^{30}\). Even for games
of much higher complexity, genetic algorithms have been used with success. While
not able to provide a complete solution, genetic algorithms have been applied to
some subproblems for games such as Chess~\cite{Mitsuta:2010:OPG:1994486.1994517}
(state space complexity of \(\approx10^{50}\)) and
Go~\cite{shah2012Go,Blackman2009Go} (state space complexity of
\(\approx10^{172}\)). It is clear that the state space complexity of a game is
not barrier to the use of evolutionary computing.

These numbers will become important in Chapter~\ref{chap:algorithm} where the
state space complexity of Monopoly is discussed.

These deterministic, two-player, zero-sum games are widely studied because
they are relatively simple to understand and play. Many classic game theory
games like Prisoner's Dilemma have very small state spaces. Some of them have a
relatively small search space (Othello, Chess and Go being obvious exceptions),
which makes them tractable to study and simulation. However, there
is a whole class of games with many examples that are non-deterministic,
multi-player (but still usually zero-sum) games.

\section{Non-Deterministic games}

Non-deterministic games are those that have some element of randomness in them.
Many gambling games such as poker, blackjack, roulette, or craps are based
on random outcomes. Randomness in those games is provided by the shuffling of
cards, the spin of a wheel and roll of a ball, or the throw of dice. Many
recreational board games use dice or cards to provide a random element to the
game. 

While not as well studied as the two player games mentioned previously, there
has been some research on using evolutionary computation to create winning
strategies for non-deterministic games. In this section we look at two games
on which evolutionary computing has been applied: Acquire and Monopoly

\subsection{Acquire: Evolving Board Evaluation Functions for a Complex Strategy
Game}

Acquire is a board game in which players attempt to gain wealth by building and
investing in properties\footnote{See \url{http://en.wikipedia.org/wiki/Acquire}
for details}. The player with the most wealth at the end of the game is the
winner.

Anthony applied genetic programming (a subset of evolutionary computing) to the
problem of evolving board evaluation functions for the game~\cite{Anthony2002}.
She identified two basic decisions a board evaluation function would need to
make for the game: where to place a tile (build a property) on the board and
which property to invest in. 

Like many other games, there is no objective fitness criteria for Acquire, so
Anthony used a competitive fitness function. She evaluated several possible
methods for computing a game score including number of wins, amount of money
earned, and ratio of money earned in a game to total amount of money earned by
all players in the same game. The last method was the one used in her
experiments.

There were 1000 players in a generation, and 50 generations were evolved. Each
player in a generation played 10 games against other players in the population.
The raw score was computed for each game, and the average over 10 games was the
player's fitness for that generation. To validate that the population was
evolving better evaluation functions, the best player in the final generation
was played against the best player in each previous generation.

\subsection{Strategies for the Game of Monopoly}

The last game we look at in this section is the game of Monopoly, which is the
subject of this research. A detailed overview of the game is presented in
Chapter~\ref{chap:monopoly}, but it will be helpful to understand a few basics
about the game here. 

In the game of Monopoly, players can purchase specific locations on the game
board. Some of those locations are known as streets, and streets are further
grouped into one of eight color groups (See Figure~\ref{figure-monoboard}). When
a player lands on a location owned by another player, the first player pays rent
to the owner. If a player owns all the streets in a color group, the player can
build houses or hotels which dramatically increases the rent.

\subsubsection{Monopoly as a Markov Process}

Some of the first studies of the game were statistical analyses of which
locations were more valuable to own. Ash and Bishop modeled the Monopoly board
as a Markov process~\cite{Ash1972}. From this they concluded that the properties
with the greatest expected income where streets which comprise the Green group
on the east side of the board, followed by the two groups (Red and Yellow) on
the north side of the board.

Abbott and Richey extended Ash's and Bishop's work with a more
rigorous~\footnote{In 1972, Ash and Bishop did not have the same computing
resources to apply their calculations that Abbott and Richey had in 1997}
examination of the game as a Markov process~\cite{Abbott1997}. They came to a
similar conclusion: They expected income is greatest from the Green group,
followed by Red, Yellow, and Orange. However, because it costs much less to
build houses and hotels on the Orange group, the more valuable group is Orange,
followed by Red, then Yellow.

\subsubsection{An Evolutionary Approach to Strategies for the Game of Monopoly}

Frayn used evolutionary computing to develop a strategy for property valuation
and property management~\cite{DBLP:conf/cig/Frayn05}. Frayn used four different
chromosomes for each individual in the population. Three of the four chromosomes
represented the values of properties that could be owned by players. Each gene
in those chromosomes represented the value of a particular property. The fourth
chromosome represented other game decisions such as when to mortgage a property,
or the minimum amount of cash a player should retain.

Frayn came to the conclusion that the best property group was the Orange group.
In contrast to the conclusions of the previous two studies, the best individuals
in Frayn's experiment avoided the Red, Yellow, and Green groups. His experiment
also found that the properties on the south and west sides all had values
greater than their costs, whereas everything on the north and east sides had
values less than their costs. His individuals were thus buying the properties on
the south and west sides, and avoiding properties on the north and east sides.

Frayn used a genetic algorithm with 1000 players in a generation, and
1420 generations were evolved. He used a competitive fitness function where each
player in a generation played 100 games against three other players in the
population. The winner of the game received a score of +3 points, second place
recieved +2 points, third place received +1 point, and fourth place received 0
points. The total fitness was the sum of points over 100 games.

\subsubsection{Property Trading Strategies}

Yasumura et al., studied different negotiating strategies to identify the best
way to conduct negotiations between players~\cite{Yasumura2001Negotiate}. They
constructed an evaluation function that was the summation of property values
under different ownership conditions, less expected losses over the short term
(1 dice roll) and medium term (1 trip around the board). The summation and
losses were each also weighted by constant terms that were varied to make
strategies more bullish (more weight to potential gains and more likely to trade
a property) or bearish (more weight to expected losses and more likely not to
trade a property). They found that a strategy where the gains and losses in
their evaluation function were weighted evenly gave the best results.

\section{Competitive Fitness Functions}

When used in optimization problems, genetic algorithms usually rely on an
objective fitness function. However in the particular case of games, there is no
objective way to measure fitness, and so as seen previously in this chapter a
competitive fitness function must be used. In this section we review some of the
research into competitive fitness functions.

\subsection{Competitive Environments Evolve Better Solutions for Complex Tasks}

Angeline and Pollack conducted an early study into competitive environments for
evolutionary computing. They claimed that ``\ldots for many complex tasks an
independent fitness function is either impractical or impossible to
provide.~\cite{Angeline:1993:CEE:645513.657590}'' Their research attempted to
demonstrate that fitness functions that are dependent on the constituents of the
population ``\ldots can provide a more robust training environment than
independent fitness functions.~\cite{Angeline:1993:CEE:645513.657590}''

In their paper they discussed three different ways to structure a competitive
fitness function.

First, they looked at a round robin style of tournament where every member of
the population engages in a competition with every other member of the
population. In Axelrod's study of the evolution of cooperation in Prisoner's
Dilemma, every member of the population played every other member of
the population~\cite{Axelrod1984}. Of course, if the size of the population is
\(n\), the number of competitions is of order \(O(n^2)\), so this is not
very scalable.

Second, they looked at a bipartite competitive environment. In this environment
every individual in one population competes against some individual in another
population. This type of competition was used in a study of sorting networks by
Hillis~\cite{Hillis:1990:CPI:87498.87560}. Hillis had two populations: one
population consisted of sorting networks, and the second population consisted of
sets of test cases, where each set is an individual. Each sorting network
competed against one set of test cases. The fitness of each sorting network was
scored based on the percent of test cases that were sorted correctly; whereas
each set of test cases was scored based on the percent of test cases that were
not sorted correctly. This type of fitness environment requires \(n\)
competitions for a population of size \(n\).

Finally they looked at a single elimination tournament style of
competition within a single population. In this style of competition, pairs of
individuals are randomly selected from a population and engage in a competition.
The winner is returned to the population, the loser is temporarily removed from
the population. This proceeds until a single individual remains. The fitness of
each individual is the number of competitions that the individual won. This
requires \(n-1\) competitions for a population of size \(n\).

Angeline and Pollack found that the tournament fitness function performed well
in evolving players for the game of Tic-Tac-Toe. The tournament fitness function
has several advantages over other fitness functions: it does not require task
knowledge of the optimization problem as an objective fitness function might;
and it avoids the need to develop a separate population and fitness function as
the bipartite competition does.

\subsection{A Comparative Study of Two Competitive Fitness Functions}

Panait and Luke provided a further examination of tournament selection versus
round robin style competitions~\cite{Panait02acomparative}. In their study they
used both tournament selection and generalized round robin competition on four
different problems: two versions of the game Nim, the Rosenbrock function, and
the Rastrigin function~\footnote{Like other functions used in the study of
optimization, the Rosenbrock and Rastrigin functions have many local minima and
a single global minimum. For examples see
\url{https://en.wikipedia.org/wiki/Rosenbrock_function} and
\url{https://en.wikipedia.org/wiki/Rastrigin_function}}.

Panait and Luke presented a generalized version of round robin called K-Random
Opponents. When \(K=1\), then K-Random Opponents is a single elimination
tournament: the population is divided into pairs and each pair engages in a
single competition with the loser eliminated from further competitions in that
generation. This is used when the cost of a single competition is relatively
high and thus it would be impractical to perform a full round robin tournament.
When \(K=n-1\), then a complete round robin tournament is conducted where every
member of the population engages in one competition against every other member.
When \(1 < K < n-1\) then each individual competes against \(K\) opponents.

For each of the four problems they evaluated, Panait and Luke found that in
general, the Single Elimination Tournament performed better than K-Random
Opponents for almost any value of K. The specific cases where Single Elimination
Tournament performed worse than K-Random Opponents was when noise was introduced
into the fitness calculation. Noise was introduced by taking a random number of
competitions and exchanging the scores of the two players. That is, for a random
percentage of games, the loser of the game is awarded a win, and the winner is
given the loss. Luke and Panait found that Single Elimination Tournament
performed worse than K-Random Opponents for some values of \(K\) when the noise
percentage was greater than 30\%.

The research cited above into Checkers, Othello, Acquire, and Monopoly all used
K-Random
Opponents~\cite{Fogel2000Anaconda,journals/tec/ChellapillaF01,ChongTW05,Anthony2002,DBLP:conf/cig/Frayn05}.

\subsection{Fitnessless Coevolution}

More recently Ja\'{s}kowski, Krawiec, and Wieloch realized that when a
competitive fitness function was being used as part of a genetic algorithm, one
could skip the fitness evaluation component and use the competitions directly in the
selection phase of the algorithm~\cite{Jaskowski:2008:FC:1389095.1389161}. 

When using fitnessless coevolution, the fitness evaluation step is skipped
completely. The genetic algorithm then proceeds to the selection phase where
individuals are selected for propagation to the next generation, either
directly, or as part of reproduction. For each new individual in the next
generation, a sample set \(S\) of individuals is selected with replacement from
the current generation. The individuals in \(S\) then compete in a single
elimination tournament and the winner of the tournament becomes a candidate for
being directly added to the next generation, or becoming a parent of a new
individual in the next generation. The size \(t\) of sample set \(S\) can be
varied. If \(t=1\) then fitnessless coevolution is the same as Single
Elimination Tournament.

Ja\'{s}kowski, Krawiec, and Wieloch evaluated fitnessless coevolution against a
similar set of problems as Panait and Luke (a single version of Nim, the
Rosenbrock function, and the Rastrigin function) and additionally tested it with
Tic-Tac-Toe. They found that fitnessless coevolution was competitive with Single
Elimination Tournament and K-Random Opponents. It usually outperformed Single
Elimination Tournament, and like Single Elimination Tournament only performed
more poorly than K-Random Opponents when there was a noise percentage more than
30\%. It simplifies the competitive fitness genetic algorithm by eliminating
numerical fitness and the evaluation phase of the genetic algorithm. Also,
fitnessless coevolution works best with tasks that satisfy the transitivity
condition: in a game, if \(Player_{i}\) beats \(Player_{j}\) and \(Player_{j}\)
beats \(Player_{k}\), then \(Player_{i}\) beats \(Player_{k}\) for all i, j, k.

However, while fitnessless coevolution simplifies the genetic algorithm by
eliminating numerical fitness, this can also be a disadvantage. In many of the
studies discussed above, the best individuals in a generation were often
validated by competing them against some other benchmark player, either a
different AI opponent or a human competitor. With no numerical fitness, it is
more difficult to select the fittest players in a generation for the objective
competition.

\section{Summary}

Evolutionary computing has been applied to the study of games for many years. It
has been used successfully for the very simplest of games to the most
complicated of games. When genetic algorithms are applied to games, it is almost
always necessary to use a competitive fitness function. Researchers generally
use K-Random Opponents as the competition structure. If the game has a very high
time- or space- complexity they may use a single elimination tournament
style of competition \(K=1\). Even if the game has low complexity, in the
absence of noise, single elimination tournament style competitions appear to
provide the most robust fitness determination. However, if there is a lot of
noise, single elimination tournament does worse than K-Random Opponents with a
higher value of K. This is because noise can distort the results of a single
competition and multiple competitions are needed to smooth out the effects of
noise.